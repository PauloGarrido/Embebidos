\documentclass[10pt,a4paper]{article}
\usepackage[utf8]{inputenc}
\usepackage[spanish]{babel}
\usepackage{amsmath}
\usepackage{amsfonts}
\usepackage{amssymb}
\usepackage{graphicx}
\usepackage{multicol}
\usepackage{titling}
\usepackage{titlesec}
\usepackage{array}
\usepackage{bm}
\usepackage{afterpage}
\usepackage{float}
\usepackage{graphicx}
\usepackage{epstopdf}
\usepackage{longtable}
\usepackage{xcolor}
\usepackage{epigraph}
\usepackage{enumerate} 
\setlength\epigraphwidth{1.5\textwidth}
\usepackage{subfigure}
\usepackage{anyfontsize}
\usepackage[left=2cm,right=2cm,top=2cm,bottom=2cm]{geometry}
\usepackage[colorlinks=true,
            linkcolor=blue,
            citecolor=blue,
            urlcolor=blue]{hyperref}

\begin{document}
\author{Paulo C. Garrido-Lechón\\Cristian S. Hernández-Ramírez\\Luis A. Lima-Pambaquishpe} % CAMBIAR A AUTORES
\title{MATERIA DE SISTEMAS EMBEBIDOS\\ % TITULO  \\ ES ENTER
RESUMEN TEÓRICO}
\maketitle  
\section{Resumen} % nuevas secciones
\begin{multicols}{2} % texto en 2 columnas

Las condiciones climáticas del mundo son un sistema complejo y acoplado que cualquier alteración ambiental en algún sitio tiene impacto en todo el planeta tierra.\\
Una consecuencia alarmante del cambio climático que sucede en todo el mundo es el crecimiento del océano, el cual en los últimos años a sufrido un acrecentamiento aproximado de 20 cm., pero para el próximo siglo aumentara más rápidamente lo que las ciudades cercanas a la costa enfrentarán inundaciones y para 2050, algunas de estas ciudades necesitarán un terraplén para conservar con su vida. Es predecible que los niveles de agua en el mar suban de 30 a 120 cm lo que podrá inundar varias pequeñas Islas del Pacífico, Centros Turísticos Costeros y Ciudades Costeras.\\
Si fuera el caso de que el Iceberg de Groenlandia y el campo de hielo de la Antártida se desintegrara, el nivel del mar aumentaría en gran cantidad implicando que las alteraciones en el clima, los glaciares del mundo desaparecen, llevan a la persistencia del iceberg polar y las nevadas. Además, se crean una lluvia con variaciones climáticas por lo que generara lluvias agresivas en reemplazo de una lluvia equilibrada. 
Aun que exista fuertes lluvias en algunas áreas, el hambre y los rayos de calor de larga duración se vuelven comunes. Los estudios realizados durante los últimos años han generado evidencia con resultados negativos debido a que existe un aumento de temperatura y el número de hambruna global. \\
\\
\textbf{*Calidad del aire y efectos en la salud} - ¿Cómo puede la tecnología inalámbrica? ¿Las redes de sensores contribuyen?\\
La contaminación del aire, como resultado de una mayor industrialización, la urbanización y la movilidad individual en vehículos se ha convertido en un gran riesgo para la salud en todos los países del mundo. La Organización Mundial de la Salud (OMS) estima que una de cada ocho muertes prematuras se debe a los efectos de las partículas contaminantes del aire. 
Siguiendo estos números, la OMS recomienda límites para contaminantes nocivos como el NO2 que es límite anual: \[40\frac{\mu {g}}{m^3}
,\] Límite por hora: \[200\frac{\mu{g}}{m^3}\]
Además, los impactos en la salud resultan en un gran daño económico. Los costos anuales de los impactos en la salud de la contaminación del aire se estiman enser EUR 330 – 940. Se debe realizar para aumentar la calidad de aire (AQ).
La Vigilancia contaminación del aire ayuda de las siguientes cuatro maneras:
\begin{enumerate}[1.]
\item Se puede investigar dónde y bajo qué circunstancias produce una alta contaminación del aire, para que puedan promulgar las leyes, para reducir las emisiones.
\item Solo mediciones precisas de (AQ).
\item La conciencia ciudadana.se puede levantar. 
\item La investigación sobre la salud adversa.
\end{enumerate}
Los efectos de la contaminación del aire dependen en gran medida de datos precisos sobre contaminación del aire, para determinar a la exposición de contaminantes que están los individuos.\\
\\
\textbf{*Calidad del Aire.}\\
El último informe "Calidad del aire en Europa" de la Agencia Europea de Medio Ambiente. (EEA 2015) prevé casi cinco millones de vidas perdidas en los EU debido a las altas concentraciones de PM.
Son una estimación del promedio de años que una persona hubiera vivido si no hubiera muerto prematuramente, dando mayor peso a las muertes a una edad más temprana y menor peso a muertes a una edad más avanzada. Para los 507,4 millones de habitantes de la UE, esto significa un Pérdida promedio de más de 3 días cada año.
Además, hablando de las condiciones medias, para la calidad del aire tiene un limitado sentido. La situación suele ser peor en áreas altamente pobladas donde la mayoría población vive y, por la misma razón, la emisión de contaminantes es mayor, de hecho, el mismo informe, refiriéndose a 990 estaciones de monitoreo urbano en 736 Ciudades europeas, muestra que 202 de ellas (27,4\%) han superado el límite de 35 días por encima de 50 lg / m3 para concentraciones medias diarias de PM10.
Las concentraciones y los impactos en la salud no se conocen completamente y, por lo tanto, los límites propuestos por la Organización Mundial de la Salud son incluso más estrictos que los adoptados por Reglamentos de la UE.
La situación es bastante similar para otros contaminantes tradicionales.\\
\\
\textbf{*Internet De Las Cosas IoT.}\\
Es un modelo importante dentro del tema de las Telecomunicaciones Inalámbricas, esto muestra una Red Global de dispositivos interconectados configuradas con una dirección única y protocolos de comunicación.
El propósito fundamental de IoT es la organización de dispositivos electrónicos que recolecten e intercambien datos con el objetivo común del avance tecnológico. Además, ejecuta a estos dispositivos como receptivos, adaptables y omnipresentes en nuestras vidas.\\
Gracias a IoT surge las tecnologías de Identificación por Radiofrecuencia (RFID), Comunicación de Campo Cercano (NFC) y Red de Sensores (SN), que permiten la aplicación de los sistemas de comunicación e información en el entorno. \\
Un término primordial es el Objeto Inteligente, el cual es un mecanismo físico cibernético o un sistema integrado, cuyo elemento importante es un dispositivo físico y un componente que procesa los datos del sensor y garantiza una comunicación inalámbrica a Internet. Las características de estos mecanismos inteligentes son detectar, registrar y comprender todas las interacciones entre ellos y el medio exterior.\\
\\
\textbf{*Sensor distribuido de datos de computación en la ciudad inteligente.}\\
Nuestro entendimiento hacia la Internet de las cosas (IoT) ha estado en constante evolución. Hace unos diez años la atención se centró principalmente en la accesibilidad de las cosas basado en etiquetas RFID. 
El IoT (Internet de las cosas) fue descrito como una Red mundial de objetos interconectados. 
A lo largo de los años, hemos sido testigos de la aparición de muchas aplicaciones de IoT se describen como "inteligentes" (por ejemplo, ciudad inteligente, oficina inteligente, transporte inteligente).
Esto era el caso hasta hace poco, cuando los investigadores comenzaron a repensar sobre la pregunta "lo que realmente hace que una aplicación IoT sea inteligente.
La respuesta no es sorprendente: "datos", especialmente big data, que barre muchos de los campos de investigación en los últimos días. 
Los datos son la parte más importantes de las aplicaciones inteligentes, sin embargo, como el tamaño de los datos de IoT continúan creciendo a medida que aumenta la velocidad, no es factible transferir todos los datos sin procesar.
En un centro de datos centralizado, el procesamiento de big data, pasan por muchas tareas de reprocesamiento estándar (por ejemplo, datos limpieza, integración y abstracción) antes de realizar el análisis (por ejemplo, tareas de mapeo reducido y algoritmos de minería de datos).
Este proceso, especialmente en el reprocesamiento, requiere mucho tiempo y costo, y generalmente resulta en una alta latencia para el servicio los consumidores.
El reciente desarrollo de Fog Computing y MobileEdge Computing (MEC) permite capacidades de computación en la nube. Estas nubes de borde tienen recursos flexibles para el procesamiento de datos distribuidos. Las instalaciones proporcionan la plataforma necesaria para realizar cálculos inteligentes de manera más eficiente y permite a las aplicaciones y servicios presenten menor latencia y mejoras, con la calidad de servicios.\\
\\
\textbf{*Desarrollo de plataforma de detección para mediciones y análisis de calidad del aire.}\\
De acuerdo a la Organización Mundial de la Salud (OMS) todos los años existe alrededor de 3 millones de personas que mueren por la contaminación del aire, de las cuales la mitad son en áreas cerradas. Algunas de estas muertes incluso mayores que las por accidentes de vehículos. La deterioración del problema se debe al cambio climático o condiciones meteorológicas para la distribución de contaminación como la velocidad del viento, alta índice de humedad en el aire, la presión atmosférica, la temperatura, etc. Esta situación lleva la contaminación emitida por las industrias a empeorar la calidad del aire especialmente cuando es época de invierno. Las mediciones y el monitoreo de la calidad el aire es necesarias para el análisis de ambientes heterogéneos con diferentes fuentes de emisión como áreas urbanas. Por lo que se desarrolló un sistema efectivo de monitoreo el cual alarma cuando existen bastantes micro partículas en el aire. En la región de Prishtina se alcanzó una gran cantidad de micro partículas por metro cúbico, lo cual pasó inadvertido por la embajada de los Estados Unidos.\\
\\
\textbf{*Sensores inteligente y sistemas de libros.}\\
Los biosensores contienen ventajas que los sensores de dispositivos basados en otros principios de óptica u ópticos. Una de las principales ventajas es que tiene una fácil integración con los chips CMOS. Los sensores bio químicos utilizados para detección de las señales eléctricas detectan las cargas eléctricas o los campos eléctricos. El primer tipo se llama "sensor amperométrico", y el último es el "sensor potenciométrico". Las cargas eléctricas tienen diferentes formas dependiendo de las reacciones bioquímicas involucradas en la detección específica. A pesar de que la amperometría se usa para el sensor de glucosa comercial, la mayoría del chip biosensor eléctrico se basa en el principio FET. Las ventajas de la integración en los chips CMOS son muchas: tamaño pequeño, más inmunidad al ruido externo, potencial para la selección múltiple de las biomoléculas, por enumerar algunos.
\\
\end{multicols} %termina texto en columnas
\end{document}